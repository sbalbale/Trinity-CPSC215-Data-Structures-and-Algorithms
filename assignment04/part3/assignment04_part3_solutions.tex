\documentclass{article}
\usepackage{amsmath}
\usepackage{graphicx}
\usepackage{verbatim}

\title{Solutions to Assignment 4 - Part III: Working with Trees}
\author{Sean Balbale}
\date{\today}

\begin{document}

\maketitle
\newpage
\section{Problem 1: Pre-order and Post-order Traversal}

\textbf{Given Tree:}
\begin{verbatim}
        j
       / | \
      f  c  e
     /   /  |  \
    d   a   h   i
   /          |
  g           b
\end{verbatim}

\textbf{Pre-order Traversal (Root → Left → Right):}
\[
j, f, d, g, c, a, e, h, b, i
\]

\textbf{Post-order Traversal (Left → Right → Root):}
\[
g, d, f, a, c, b, h, i, e, j
\]

---

\section{Problem 2: Expression Tree for Polynomial}

Given polynomial:
\[
5y^2 - 3y + 2
\]

\subsection{(a) Prefix Expression (Pre-order)}
\[
+ - * 5 ^ y 2 * 3 y 2
\]

\subsection{(b) Postfix Expression (Post-order)}
\[
5 y 2 ^ * 3 y * - 2 +
\]

\textbf{Expression Tree:}
\begin{verbatim}
        (+)
       /   \
     (-)    2
    /   \
  (*)   (*)
 /   \   /   \
5   (^) 3   y
    /   \
   y     2
\end{verbatim}

---

\section{Problem 3: Constructing a Binary Tree}

\textbf{Given Traversals:}
- Preorder: \texttt{EXAMFUN}
- Inorder: \texttt{MAFXUEN}

\textbf{Constructed Tree:}
\begin{verbatim}
        E
       /  \
      A    X
     /    /  \
    M    U    N
     \
      F
\end{verbatim}

---

\section{Problem 4: Binary Search Tree Traversals}

\textbf{Given BST:}
\begin{verbatim}
        100
       /   \
     20     200
    /  \    /   \
  10   30  150   300
\end{verbatim}

\textbf{Pre-order Traversal:}  
\[
100, 20, 10, 30, 200, 150, 300
\]

\textbf{In-order Traversal (Sorted Order):}  
\[
10, 20, 30, 100, 150, 200, 300
\]

\textbf{Post-order Traversal:}  
\[
10, 30, 20, 150, 300, 200, 100
\]

---

\section{Problem 5: Transforming a Tree into a BST}

\textbf{Given Tree:}
\begin{verbatim}
        "easy"
       /       \
   "in"     "november"
      \       /         \
    "can"  "be"     "quizzes"
           /
        "fun"
\end{verbatim}

\textbf{(a) Pre-order Traversal:}  
\[
\text{"easy", "in", "can", "november", "be", "fun", "quizzes"}
\]

\textbf{(b) Post-order Traversal:}  
\[
\text{"can", "in", "fun", "be", "quizzes", "november", "easy"}
\]

\textbf{(c) Balanced BST:}
\begin{verbatim}
        "easy"
       /       \
   "can"     "november"
   /           /        \
"be"      "in"   "quizzes"
           /
        "fun"
\end{verbatim}

---

\section{Problem 6: Constructing a Binary Tree from Traversals}

\textbf{Given Traversals:}
- In-order: \texttt{DBEAFC}
- Pre-order: \texttt{ABDECF}

\textbf{Constructed Tree:}
\begin{verbatim}
      A
     /  \
    B    C
   / \    \
  D   E    F
\end{verbatim}

\textbf{Post-order Traversal:}  
\[
D, E, B, F, C, A
\]

\textbf{Possible BSTs of Height 2:}
\begin{verbatim}
      C
     /   \
    B     E
   / \   / \
  A   D F
\end{verbatim}

---

\section{Problem 7: Preorder vs. Postorder Relationship}

\textbf{Can preorder and postorder be the same?}  
No, unless it is a single-node tree.

\textbf{Can preorder be the reverse of postorder?}  
Yes, in a completely linear tree:
\begin{verbatim}
    A
     \
      B
       \
        C
         \
          D
\end{verbatim}
where \textbf{Preorder} = `A B C D` and \textbf{Postorder} = `D C B A`.

---

\section{Problem 8: General Tree Representation as a Binary Tree}

\textbf{Given General Tree \( T \) and Converted Binary Tree \( T' \):}

\begin{verbatim}
(a) General Tree T:
        A
       / | \
      B  C  D
     / \
    E   F
       
(b) Binary Tree T':
        A
       /
      B
     /  \
    E    C
     \     \
      F     D
\end{verbatim}

\subsection{(a) Is a preorder traversal of \( T' \) equivalent to a preorder traversal of \( T \)?}  
Yes.

\subsection{(b) Is a postorder traversal of \( T' \) equivalent to a postorder traversal of \( T \)?}  
Yes.

\subsection{(c) Is an inorder traversal of \( T' \) equivalent to a standard traversal of \( T \)?}  
No.

\end{document}
