\documentclass{article}
\usepackage{amsmath, amssymb}
\usepackage{booktabs}
\usepackage{enumitem}
\usepackage[margin=1in]{geometry}
\begin{document}

\title{Part II: Hashing and Hash Tables}
\author{Sean Balbale}
\date{}
\maketitle

\section*{Part (a): Quadratic Probing}

We are given a 10-slot closed hash table (slots 0 through 9) and the hash function 
\[
h(k)= k \mod 10.
\]
We wish to insert the following keys using quadratic probing:
\[
5,\quad 12,\quad 8,\quad 20,\quad 14,\quad 7.
\]
Quadratic probing uses the probe sequence:
\[
h(k),\quad h(k)+1^2,\quad h(k)+2^2,\quad h(k)+3^2,\quad \dots \quad (\text{all mod }10).
\]
However, notice that for these keys the initial hash locations are all distinct:
\begin{enumerate}[label=\textbf{Step \arabic*:}, leftmargin=1.8cm]
    \item \textbf{Insert 5:} 
    \[
    h(5)=5 \mod 10 = 5.
    \]
    Slot 5 is empty, so place 5 in slot 5.
    \item \textbf{Insert 12:} 
    \[
    h(12)=12 \mod 10 = 2.
    \]
    Slot 2 is empty, so place 12 in slot 2.
    \item \textbf{Insert 8:} 
    \[
    h(8)=8 \mod 10 = 8.
    \]
    Slot 8 is empty, so place 8 in slot 8.
    \item \textbf{Insert 20:} 
    \[
    h(20)=20 \mod 10 = 0.
    \]
    Slot 0 is empty, so place 20 in slot 0.
    \item \textbf{Insert 14:} 
    \[
    h(14)=14 \mod 10 = 4.
    \]
    Slot 4 is empty, so place 14 in slot 4.
    \item \textbf{Insert 7:} 
    \[
    h(7)=7 \mod 10 = 7.
    \]
    Slot 7 is empty, so place 7 in slot 7.
\end{enumerate}

Thus, the final hash table is:

\bigskip
\begin{center}
\begin{tabular}{c|c}
\textbf{Slot} & \textbf{Key} \\ \midrule
0 & 20 \\
1 & --- \\
2 & 12 \\
3 & --- \\
4 & 14 \\
5 & 5 \\
6 & --- \\
7 & 7 \\
8 & 8 \\
9 & --- \\
\end{tabular}
\end{center}
\bigskip

\section*{Part (b): Double Hashing}

Now we use a 10-slot hash table with the following two hash functions:
\[
\text{Primary: } h(k)= k \mod 10,\quad \text{Secondary: } h_2(k)=7 - (k \mod 7).
\]
The keys to insert are the same: 
\[
5,\quad 12,\quad 8,\quad 20,\quad 14,\quad 7.
\]
Double hashing uses the probe sequence:
\[
h(k),\quad h(k)+h_2(k),\quad h(k)+2\cdot h_2(k),\quad \dots \quad (\text{all mod }10).
\]
We insert the keys as follows:
\begin{enumerate}[label=\textbf{Step \arabic*:}, leftmargin=1.8cm]
    \item \textbf{Insert 5:} 
    \[
    h(5)=5 \mod 10 = 5.
    \]
    Slot 5 is empty; insert 5.
    \item \textbf{Insert 12:} 
    \[
    h(12)=12 \mod 10 = 2.
    \]
    Slot 2 is empty; insert 12.
    \item \textbf{Insert 8:} 
    \[
    h(8)=8 \mod 10 = 8.
    \]
    Slot 8 is empty; insert 8.
    \item \textbf{Insert 20:} 
    \[
    h(20)=20 \mod 10 = 0.
    \]
    Slot 0 is empty; insert 20.
    \item \textbf{Insert 14:} 
    \[
    h(14)=14 \mod 10 = 4.
    \]
    Slot 4 is empty; insert 14.
    \item \textbf{Insert 7:} 
    \[
    h(7)=7 \mod 10 = 7.
    \]
    Slot 7 is empty; insert 7.
\end{enumerate}

No collisions occurred in the primary hash positions; hence the final table is identical to part (a):

\bigskip
\begin{center}
\begin{tabular}{c|c}
\textbf{Slot} & \textbf{Key} \\ \midrule
0 & 20 \\
1 & --- \\
2 & 12 \\
3 & --- \\
4 & 14 \\
5 & 5 \\
6 & --- \\
7 & 7 \\
8 & 8 \\
9 & --- \\
\end{tabular}
\end{center}
\bigskip

\section*{Part (c): Double Hashing with Custom Hash Functions in a 13-Slot Table}

We now have a 13-slot hash table (slots 0 through 12). The keys to insert are:
\[
2,\quad 8,\quad 31,\quad 20,\quad 19,\quad 18,\quad 53,\quad 27.
\]
The hash functions are defined as:
\[
H_1(k)= k \mod 13,
\]
\[
H_2(k)= \text{Rev}(k+1) \mod 11,
\]
where $\text{Rev}(n)$ reverses the decimal digits of $n$. (For example, $\text{Rev}(37)=73$, and $\text{Rev}(7)=7$.)

\begin{enumerate}[label=\textbf{Step \arabic*:}, leftmargin=1.8cm]
    \item \textbf{Key 2:} 
    \[
    H_1(2)=2 \mod 13 = 2.
    \]
    Compute 
    \[
    H_2(2)=\text{Rev}(2+1)=\text{Rev}(3)=3 \quad \Rightarrow\quad 3 \mod 11=3.
    \]
    Slot 2 is empty; insert 2 at slot 2.
    
    \item \textbf{Key 8:} 
    \[
    H_1(8)=8 \mod 13 = 8.
    \]
    Compute 
    \[
    H_2(8)=\text{Rev}(8+1)=\text{Rev}(9)=9 \quad \Rightarrow\quad 9 \mod 11=9.
    \]
    Slot 8 is empty; insert 8 at slot 8.
    
    \item \textbf{Key 31:} 
    \[
    H_1(31)=31 \mod 13 = 5.
    \]
    Compute 
    \[
    H_2(31)=\text{Rev}(31+1)=\text{Rev}(32)=23,\quad 23 \mod 11 = 1.
    \]
    Slot 5 is empty; insert 31 at slot 5.
    
    \item \textbf{Key 20:} 
    \[
    H_1(20)=20 \mod 13 = 7.
    \]
    Compute 
    \[
    H_2(20)=\text{Rev}(20+1)=\text{Rev}(21)=12,\quad 12 \mod 11 = 1.
    \]
    Slot 7 is empty; insert 20 at slot 7.
    
    \item \textbf{Key 19:} 
    \[
    H_1(19)=19 \mod 13 = 6.
    \]
    Compute 
    \[
    H_2(19)=\text{Rev}(19+1)=\text{Rev}(20)=2,\quad 2 \mod 11=2.
    \]
    Slot 6 is empty; insert 19 at slot 6.
    
    \item \textbf{Key 18:} 
    \[
    H_1(18)=18 \mod 13 = 5.
    \]
    Slot 5 is already occupied (with 31). Now compute 
    \[
    H_2(18)=\text{Rev}(18+1)=\text{Rev}(19)=91,\quad 91 \mod 11 = 91-88=3.
    \]
    Use the double hashing probe sequence:
    \begin{itemize}
        \item First probe: index \(5\) (occupied).
        \item Second probe: index \( (5+3) \mod 13 = 8\) (occupied).
        \item Third probe: index \( (5+2\cdot3) \mod 13 = (5+6)=11\) (empty).
    \end{itemize}
    Insert 18 at slot 11.
    
    \item \textbf{Key 53:} 
    \[
    H_1(53)=53 \mod 13 = 1.
    \]
    Compute 
    \[
    H_2(53)=\text{Rev}(53+1)=\text{Rev}(54)=45,\quad 45 \mod 11 = 1.
    \]
    Slot 1 is empty; insert 53 at slot 1.
    
    \item \textbf{Key 27:} 
    \[
    H_1(27)=27 \mod 13 = 1.
    \]
    Slot 1 is occupied (by 53). Now compute 
    \[
    H_2(27)=\text{Rev}(27+1)=\text{Rev}(28)=82,\quad 82 \mod 11 = 82-77=5.
    \]
    Probe sequence:
    \begin{itemize}
        \item First probe: index \(1\) (occupied).
        \item Second probe: index \( (1+5) \mod 13 = 6\) (occupied).
        \item Third probe: index \( (1+2\cdot5) \mod 13 = 11\) (occupied).
        \item Fourth probe: index \( (1+3\cdot5) \mod 13 = 16 \mod 13 = 3\) (empty).
    \end{itemize}
    Insert 27 at slot 3.
\end{enumerate}

The final hash table is:

\bigskip
\begin{center}
\begin{tabular}{c|c}
\textbf{Slot} & \textbf{Key} \\ \midrule
0  & --- \\
1  & 53 \\
2  & 2 \\
3  & 27 \\
4  & --- \\
5  & 31 \\
6  & 19 \\
7  & 20 \\
8  & 8 \\
9  & --- \\
10 & --- \\
11 & 18 \\
12 & --- \\
\end{tabular}
\end{center}
\bigskip

\section*{Part (d): Pseudo-Random Probing}

We now use a 10-slot hash table with the hash function 
\[
h(k)= k \mod 10,
\]
and pseudo-random probing using the permutation of offsets:
\[
5,\quad 9,\quad 2,\quad 1,\quad 4,\quad 8,\quad 6,\quad 3,\quad 7.
\]
The keys to insert are:
\[
3,\quad 12,\quad 9,\quad 2,\quad 79,\quad 44.
\]

\subsection*{Insertion Steps}

\begin{enumerate}[label=\textbf{Step \arabic*:}, leftmargin=1.8cm]
    \item \textbf{Insert 3:} 
    \[
    h(3)=3 \mod 10 = 3.
    \]
    Slot 3 is empty; insert 3 in slot 3.
    
    \item \textbf{Insert 12:} 
    \[
    h(12)=12 \mod 10 = 2.
    \]
    Slot 2 is empty; insert 12 in slot 2.
    
    \item \textbf{Insert 9:} 
    \[
    h(9)=9 \mod 10 = 9.
    \]
    Slot 9 is empty; insert 9 in slot 9.
    
    \item \textbf{Insert 2:} 
    \[
    h(2)=2 \mod 10 = 2,
    \]
    but slot 2 is occupied (by 12). Apply the pseudo-random probe offsets:
    \begin{itemize}
        \item First offset: \(5\). New index: \((2+5) \mod 10 = 7\). Slot 7 is empty; insert 2 at slot 7.
    \end{itemize}
    
    \item \textbf{Insert 79:} 
    \[
    h(79)=79 \mod 10 = 9,
    \]
    but slot 9 is occupied (by 9). Probe:
    \begin{itemize}
        \item First offset: \(5\). New index: \((9+5) \mod 10 = 4\). Slot 4 is empty; insert 79 at slot 4.
    \end{itemize}
    
    \item \textbf{Insert 44:} 
    \[
    h(44)=44 \mod 10 = 4,
    \]
    but slot 4 is occupied (by 79). Probe:
    \begin{itemize}
        \item First offset: \(5\). New index: \((4+5) \mod 10 = 9\) (occupied).
        \item Second offset: \(9\). New index: \((4+9) \mod 10 = 13 \mod 10 = 3\) (occupied).
        \item Third offset: \(2\). New index: \((4+2) \mod 10 = 6\) (empty).
    \end{itemize}
    Insert 44 at slot 6.
\end{enumerate}

The final hash table is:

\bigskip
\begin{center}
\begin{tabular}{c|c}
\textbf{Slot} & \textbf{Key} \\ \midrule
0 & --- \\
1 & --- \\
2 & 12 \\
3 & 3 \\
4 & 79 \\
5 & --- \\
6 & 44 \\
7 & 2 \\
8 & --- \\
9 & 9 \\
\end{tabular}
\end{center}
\bigskip

\subsection*{Probability That an Empty Slot Will Be Next Filled}

After inserting 44, the empty slots are: 0, 1, 5, and 8. Assume that for the next key the initial hash value, \(r=h(\text{new key})\), is uniformly distributed over \(\{0,1,\ldots,9\}\). The probing sequence for a new key (after its hash \(r\)) is:
\[
r,\quad r+5,\quad r+9,\quad r+2,\quad r+1,\quad r+4,\quad r+8,\quad r+6,\quad r+3,\quad r+7\quad (\text{mod }10).
\]

We analyze, for each starting \(r\), which empty slot is encountered first (given our table occupancy):

\begin{itemize}
    \item \textbf{\(r=0\):} Sequence: \(0\) (empty) \(\to\) \textbf{slot 0} is chosen.
    \item \textbf{\(r=1\):} Sequence: \(1\) (empty) \(\to\) \textbf{slot 1} is chosen.
    \item \textbf{\(r=2\):} Sequence: \(2\) (filled), \(2+5=7\) (filled), \(2+9=11 \mod 10=1\) (empty) \(\to\) \textbf{slot 1}.
    \item \textbf{\(r=3\):} Sequence: \(3\) (filled), \(3+5=8\) (empty) \(\to\) \textbf{slot 8}.
    \item \textbf{\(r=4\):} Sequence: \(4\) (filled), \(4+5=9\) (filled), \(4+9=13 \mod 10=3\) (filled), \(4+2=6\) (filled), \(4+1=5\) (empty) \(\to\) \textbf{slot 5}.
    \item \textbf{\(r=5\):} Sequence: \(5\) (empty) \(\to\) \textbf{slot 5}.
    \item \textbf{\(r=6\):} Sequence: \(6\) (filled), \(6+5=11 \mod 10=1\) (empty) \(\to\) \textbf{slot 1}.
    \item \textbf{\(r=7\):} Sequence: \(7\) (filled), \(7+5=12 \mod 10=2\) (filled), \(7+9=16 \mod 10=6\) (filled), \(7+2=9\) (filled), \(7+1=8\) (empty) \(\to\) \textbf{slot 8}.
    \item \textbf{\(r=8\):} Sequence: \(8\) (empty) \(\to\) \textbf{slot 8}.
    \item \textbf{\(r=9\):} Sequence: \(9\) (filled), \(9+5=14 \mod 10=4\) (filled), \(9+9=18 \mod 10=8\) (empty) \(\to\) \textbf{slot 8}.
\end{itemize}

Since the initial hash \(r\) is uniformly distributed, each value occurs with probability \(1/10\). We tally the outcomes:
\begin{itemize}
    \item \(\textbf{Slot 0:}\) Occurs when \(r=0\) \(\Rightarrow\) probability \(= \frac{1}{10}=0.1.\)
    \item \(\textbf{Slot 1:}\) Occurs when \(r=1,2,6\) \(\Rightarrow\) probability \(= \frac{3}{10}=0.3.\)
    \item \(\textbf{Slot 5:}\) Occurs when \(r=4,5\) \(\Rightarrow\) probability \(= \frac{2}{10}=0.2.\)
    \item \(\textbf{Slot 8:}\) Occurs when \(r=3,7,8,9\) \(\Rightarrow\) probability \(= \frac{4}{10}=0.4.\)
\end{itemize}


\end{document}
