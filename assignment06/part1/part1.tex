\documentclass{article}
\usepackage[utf8]{inputenc}
\usepackage{amsmath}
\usepackage{amssymb}
\usepackage{array}

\begin{document}

\section*{Part I: Huffman Encoding}

\subsection*{Given Frequencies}
\[
\begin{array}{c|c}
\text{Character} & \text{Frequency} \\
\hline
e & 34 \\
r & 22 \\
s & 24 \\
t & 28 \\
n & 15 \\
l & 10 \\
i & 9  \\
z & 8  \\
\end{array}
\]

\subsection*{Huffman Tree Construction}

\begin{enumerate}
  \item Combine $z(8)$ and $i(9)$ to form node $A(17)$.
  \item Combine $l(10)$ and $n(15)$ to form node $B(25)$.
  \item Combine $A(17)$ and $r(22)$ to form node $C(39)$.
  \item Combine $s(24)$ and $B(25)$ to form node $D(49)$.
  \item Combine $t(28)$ and $e(34)$ to form node $E(62)$.
  \item Combine $C(39)$ and $D(49)$ to form node $F(88)$.
  \item Combine $E(62)$ and $F(88)$ to form root $G(150)$.
\end{enumerate}

\subsection*{Huffman Codes}
Assigning 0 and 1 down each branch, we get:

\[
\begin{array}{c|c}
\text{Character} & \text{Code} \\
\hline
t & 00 \\
e & 01 \\
z & 1000 \\
i & 1001 \\
r & 101 \\
s & 110 \\
l & 1110 \\
n & 1111 \\
\end{array}
\]

\subsection*{Total Bits vs.\ ASCII}

\paragraph{Huffman-Encoded Bits.}
\[
\begin{aligned}
& e(34)\times2 = 68, \quad
r(22)\times3 = 66, \quad
s(24)\times3 = 72, \\
& t(28)\times2 = 56, \quad
n(15)\times4 = 60, \quad
l(10)\times4 = 40, \\
& i(9)\times4 = 36, \quad
z(8)\times4 = 32. \\
& \text{Total Huffman bits } = 68 + 66 + 72 + 56 + 60 + 40 + 36 + 32 = 430.
\end{aligned}
\]

\paragraph{ASCII Bits.}
Each character would use $8$ bits, so for $150$ characters:
\[
150 \times 8 = 1200 \text{ bits}.
\]

\paragraph{Bits Saved.}
\[
1200 - 430 = 770 \text{ bits saved in total}.
\]

\subsection*{Encoding Sample Words}

\begin{itemize}
  \item \textbf{``next''}:
  \[
  n = 1111,\quad e = 01,\quad x\ (\text{or }z)=1000,\quad t=00
  \]
  \[
  \text{``next''} = 1111\,01\,1000\,00 = 111101100000.
  \]

  \item \textbf{``stern''}:
  \[
  s = 110,\quad t = 00,\quad e = 01,\quad r = 101,\quad n = 1111
  \]
  \[
  \text{``stern''} = 110\,00\,01\,101\,1111 = 11000011011111.
  \]

  \item \textbf{``nertzrents''}:
  \[
  n=1111,\ e=01,\ r=101,\ t=00,\ z=1000,\ r=101,\ e=01,\ n=1111,\ t=00,\ s=110
  \]
  \[
  \text{``nertzrents''} = 1111\,01\,101\,00\,1000\,101\,01\,1111\,00\,110
  \]
  \[
  = 11110110100100010101111100110.
  \]
\end{itemize}

\end{document}
