\documentclass{article}
\usepackage{amsmath}

\begin{document}

\title{Asymptotic Complexity Proofs}
\author{}
\date{}
\maketitle

\section*{Part II: Asymptotic Notation Proofs}

\subsection*{1. Finding the Complexity of the Given Equation}
\textbf{Equation:}
\[
10n^3 + 24n^2 + 3n \log n + 144
\]
\textbf{Analysis:}
\begin{itemize}
    \item The highest order term dominates the function as \( n \) grows large.
    \item The dominant term is \( 10n^3 \).
\end{itemize}
\textbf{Conclusion:}
\[
10n^3 + 24n^2 + 3n \log n + 144 = \Theta(n^3)
\]

\subsection*{2. Proof: \( n^3 + 20n + 1 = O(n^3) \)}

\textbf{Definition of Big-O:}  
A function \( f(n) \) is \( O(g(n)) \) if there exist positive constants \( c \) and \( n_0 \) such that:

\[
f(n) \leq c \cdot g(n) \quad \text{for all } n \geq n_0
\]

\textbf{Step-by-Step Analysis:}
\[
f(n) = n^3 + 20n + 1, \quad g(n) = \frac{n^3}{6}
\]

For large \( n \), we approximate:
\[
n^3 + 20n + 1 \leq c \cdot \frac{n^3}{6}
\]

Ignoring lower-order terms:
\[
n^3 \leq c \cdot \frac{n^3}{6}
\]

Choosing \( c = 6 \), we get:
\[
n^3 \leq 6 \cdot \frac{n^3}{6} = n^3
\]

Thus, the inequality holds for all \( n \geq 1 \), proving:

\[
n^3 + 20n + 1 = O(n^3)
\]

\subsection*{3. Proof: \( 7n^2 + 5 = O(n^3) \)}

We need to show that:
\[
7n^2 + 5 \leq c \cdot n^3
\]

For large \( n \), the dominant term is \( 7n^2 \), so:
\[
7n^2 \leq c \cdot n^3
\]

Dividing by \( n^2 \):
\[
7 \leq c \cdot n
\]

Choosing \( c = 7 \) and \( n_0 = 1 \), the inequality holds.

\textbf{Conclusion:}
\[
7n^2 + 5 = O(n^3)
\]

\subsection*{4. Proof: \( n^3 + 20n = \Omega(n^2) \)}

\textbf{Definition of Big-Omega:}  
A function \( f(n) \) is \( \Omega(g(n)) \) if there exist positive constants \( c \) and \( n_0 \) such that:

\[
f(n) \geq c \cdot g(n) \quad \text{for all } n \geq n_0
\]

\textbf{Step-by-Step Analysis:}
\[
f(n) = n^3 + 20n, \quad g(n) = n^2
\]

For large \( n \), we approximate:
\[
n^3 + 20n \geq c \cdot n^2
\]

Dividing by \( n^2 \):
\[
n + \frac{20}{n} \geq c
\]

For large \( n \), \( \frac{20}{n} \to 0 \), so we approximate:
\[
n \geq c
\]

Choosing \( c = 1 \) and \( n_0 = 1 \), the inequality holds.

\textbf{Conclusion:}
\[
n^3 + 20n = \Omega(n^2)
\]

\subsection*{5. Proof: \( 3n \log n + 4n + 5n \log n = \Theta(n \log n) \)}

We rewrite:
\[
3n \log n + 5n \log n + 4n = 8n \log n + 4n
\]

For large \( n \), the dominant term is \( 8n \log n \), ignoring \( 4n \):

\[
8n \log n
\]

Since \( 8n \log n \) is a constant multiple of \( n \log n \), we conclude:

\[
\Theta(n \log n)
\]

\textbf{Conclusion:}
\[
3n \log n + 4n + 5n \log n = \Theta(n \log n)
\]

\end{document}
