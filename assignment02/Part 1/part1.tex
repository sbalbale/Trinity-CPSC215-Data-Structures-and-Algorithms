\documentclass{article}
\usepackage{amsmath}
\usepackage{listings}

\begin{document}

\title{Analysis of Code Running Time using Big-O Notation}
\author{}
\date{}
\maketitle

\section*{Part I: Analyzing Code for Running Time}

\subsection*{A.}
\begin{lstlisting}
int array_sum(int[] a, int n) {
    int i;
    int sum = 0;
    for (i = 0; i < n; i++) {
        sum = sum + a[i];
    }
    return sum;
}
\end{lstlisting}

\textbf{Analysis:}
\begin{itemize}
  \item The loop runs from \( i = 0 \) to \( i < n \), iterating \( n \) times.
  \item Each iteration involves a constant-time operation.
\end{itemize}
\textbf{Time Complexity:} \( O(n) \) (Linear time)

\subsection*{B.}
\begin{lstlisting}
int sum = 0;
for (i = 0; i < n; i++) {
    for (j = 0; j < n; j++) {
        for (k = 0; k < n; k++) {
            if (i == j && j == k) {
                for (l = 0; l < n * n * n; l++) {
                    sum = i + j + k + l;
                }
            }
        }
    }
}
\end{lstlisting}

\textbf{Analysis:}
\begin{itemize}
  \item The first three loops iterate \( n \) times each, contributing \( O(n^3) \).
  \item The condition \( i == j == k \) holds for \( O(n) \) cases.
  \item Inside the condition, the innermost loop runs \( O(n^3) \) times.
  \item Final complexity: \( O(n^3) \times O(n^3) = O(n^6) \).
\end{itemize}
\textbf{Time Complexity:} \( O(n^6) \) (Polynomial time)

\subsection*{C.}
\begin{lstlisting}
int sum(int a, int b) {
    int c = a + b;
    return c;
}
\end{lstlisting}

\textbf{Analysis:}
\begin{itemize}
  \item This function simply performs an addition and return, both of which take constant time.
\end{itemize}
\textbf{Time Complexity:} \( O(1) \) (Constant time)

\subsection*{D.}
\begin{lstlisting}
void fun(int n) {
    int i, j;
    for (i = 1; i <= n; i++)
        for (j = 1; j < log(i); j++)
            System.out.println("CPSC 215");
}
\end{lstlisting}

\textbf{Analysis:}
\begin{itemize}
  \item The outer loop runs \( n \) times.
  \item The inner loop runs \( O(\log i) \) times for each \( i \).
  \item Summing up over all \( i \):
    \[
      \sum_{i=1}^{n} O(\log i) = O(n \log n)
    \]
\end{itemize}
\textbf{Time Complexity:} \( O(n \log n) \)

\subsection*{E.}
\begin{lstlisting}
int binarySearch(int[] arr, int target) {
    int left = 0;
    int right = arr.length - 1;

    while (left <= right) {
        int mid = left + (right - left) / 2;

        if (arr[mid] == target) {
            return mid;
        } else if (arr[mid] < target) {
            left = mid + 1;
        } else {
            right = mid - 1;
        }
    }
    return -1;
}
\end{lstlisting}

\textbf{Analysis:}
\begin{itemize}
  \item Binary search repeatedly halves the search space.
  \item Each iteration reduces the problem size by a factor of 2.
\end{itemize}
\textbf{Time Complexity:} \( O(\log n) \) (Logarithmic time)

\end{document}
